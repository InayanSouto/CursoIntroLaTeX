% Caixas das Dicas (pacote tcolorbox)
% Material Colors
\newtcolorbox{marker}[1][]{enhanced,
  before skip=10mm,after skip=10mm,
  boxrule=0.4pt,left=5mm,right=2mm,top=1mm,bottom=1mm,
  colback=MaterialAmberA100,
  colframe=MaterialGrey900,
  sharp corners,rounded corners=southeast,arc is angular,arc=3mm,
  underlay={%
    \path[fill=tcbcol@back!80!black] ([yshift=3mm]interior.south east)--++(-0.4,-0.1)--++(0.1,-0.2);
    \path[draw=tcbcol@frame,shorten <=-0.05mm,shorten >=-0.05mm] ([yshift=3mm]interior.south east)--++(-0.4,-0.1)--++(0.1,-0.2);
    \path[fill=MaterialYellow100!50!black,draw=none] (interior.south west) rectangle node[white]{\Huge\bfseries !} ([xshift=4mm]interior.north west);
    },
  drop fuzzy shadow,#1}
  
% Caixas dos Exemplos (pacote tcolorbox)
% Material Colors
\tcbset{
    texexp/.style={
        colframe=MaterialBlue900,
        colback=white,
        coltitle=white,
        fonttitle=\small\sffamily\bfseries
        fontupper=\small, 
        fontlower=\small},
        example/.style 2 args={texexp,
        title={Exemplo \thetcbcounter: #1},label={#2}},
    }
    \newtcblisting{texexp}[1]{texexp,#1}
    \newtcblisting[auto counter,number within=section]{texexptitled}[3][]{%
        example={#2}{#3},#1}
    \newtcolorbox[use counter from=texexptitled]{texexptitledspec}[3][]{%
        example={#2}{#3},#1}

% Caixas dos Exercícios e Soluções (pacote tcolorbox)  
% Material Colors
\NewTColorBox[auto counter,number within=chapter]{exercise}{m+O{}}{%
    enhanced,
    colframe=MaterialGreen900,
    colback=white,
    coltitle=white,
    fonttitle=\bfseries,
    underlay={\begin{tcbclipinterior}
        \shade[inner MaterialGreen900,outer color=white]
            (interior.north west) circle (2cm);
        \draw[help lines,step=5mm,MaterialGreen900,shift={(interior.north west)}]
            (interior.south west) grid (interior.north east);
        \end{tcbclipinterior}},
    title={Exercício~ \thetcbcounter:},
    label={exercise:#1},
    attach title to upper=\quad,
    after upper={\par\hfill\textcolor{white}%
        {\itshape Solução na página~\pageref{solution:#1}}},
    lowerbox=ignored,
    savelowerto=solutions/exercise-\thetcbcounter.tex,
    record={\string\solution{#1}{solutions/exercise-\thetcbcounter.tex}},
    #2
}

\newtcolorbox[auto counter,number within=section,list inside=exam]{texercise}[2][]{%
    texercisestyle,
    listing file={solutions/texercise\thetcbcounter.tex},
    label={exe:#2},
    record={\string\processsol{solutions/texercise\thetcbcounter.tex}{#2}},
    title={Exercício \thetcbcounter\hfill\mdseries Solução na página \pageref{sol:#2}},
    list text={Exercício com solução na página \pageref{sol:#2}},#1}
    
\tcbset{texercisestyle/.style={arc=0.5mm, colframe=MaterialGreen900,
    colback=white, coltitle=white,
    fonttitle=\small\sffamily\bfseries, fontupper=\small, fontlower=\small,
    listing options={style=tcblatex,texcsstyle=*\color{MaterialRed900}},
}}

% Remove o comando style=tcblatex para funcionar com os compiladores XeLaTeX e LuaTex
%\tcbset{texercisestyle/.style={arc=0.5mm, colframe=MaterialOrange,
%    colback=white, coltitle=white,
%    fonttitle=\small\sffamily\bfseries, fontupper=\small, fontlower=\small,
%    listing options={texcsstyle=*\color{white}},
%}}

% \usepackage{hyperref} % for phantomlabel
\newtcbinputlisting{\processsol}[2]{%
    texercisestyle,
    listing only,
    listing file={#1},
    phantomlabel={sol:#2},%
    title={Solução do Exercício \ref{exe:#2} na página \pageref{exe:#2}},
}

% Caixa dos Comandos (pacote tcolorbox)
% Material Colors
\newtcblisting{commandshell}{colback=MaterialBlueGrey900,colupper=MaterialBlueGrey50,colframe=MaterialBlueGrey900!75!MaterialBlueGrey900,listing only,listing options={style=tcblatex,language=sh},
    every listing line={\textcolor{MaterialRed500}{\small\ttfamily\bfseries \$ }}}

% Remove o comando tcblatex para funcionar com o compiladores XeLaTeX e LuaLaTeX
%\newtcblisting{commandshell}{colback=MaterialBlueGrey900,colupper=MaterialBlueGrey50,colframe=MaterialBlueGrey900!75!MaterialBlueGrey900,listing only,listing options={language=sh},
%    every listing line={\textcolor{MaterialRed500}{\small\ttfamily\bfseries \$ }}}

% Setenças individuais Loren Lipsum (pacote lipsum)
%https://tex.stackexchange.com/questions/254901/one-sentence-of-dummy-text
% store a big set of sentences
\unpacklipsum[1-100] % it was \UnpackLipsum before version 2.0
\ExplSyntaxOn
% unpack \lipsumexp
\seq_new:N \g_lipsum_sentences_seq
\cs_generate_variant:Nn \seq_set_split:Nnn { NnV }
\seq_gset_split:NnV \g_lipsum_sentences_seq {.~} \lipsumexp

\NewDocumentCommand{\lipsumsentence}{>{\SplitArgument{1}{-}}O{1-7}}
 {
  \lipsumsentenceaux #1
 }
\NewDocumentCommand{\lipsumsentenceaux}{mm}
 {
  \IfNoValueTF { #2 }
   {
    \seq_item:Nn \g_lipsum_sentences_seq { #1 }.~
   }
   {
    \int_step_inline:nnnn { #1 } { 1 } { #2 }
     {
      \seq_item:Nn \g_lipsum_sentences_seq { ##1 }.~
     }
   }
 }
\ExplSyntaxOff