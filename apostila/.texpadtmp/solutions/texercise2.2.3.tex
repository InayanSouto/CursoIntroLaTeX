\begin{tabular}{|p{3cm}|p{3cm}|p{3cm}|p{3cm}|}\hline
\multicolumn{4}{|c|}{\bfseries\itshape Das alte Italien}\\\hline
\multicolumn{2}{|c|}{\bfseries Antike} &
\multicolumn{2}{c|}{\bfseries Mittelalter}\\\hline
\multicolumn{1}{|c|}{\itshape Republik}&
\multicolumn{1}{c|}{\itshape Kaiserreich}&
\multicolumn{1}{c|}{\itshape Franken}&
\multicolumn{1}{c|}{\itshape Teilstaaten}\\\hline
In den Zeiten der r\"{o}mischen Republik standen dem Staat jeweils zweiKonsuln vor, deren Machtbefugnisse identisch waren. &
Das r\"{o}mische Kaiserreich wurde von einem Alleinherrscher, dem Kaiser,regiert.&
In der V\"{o}lkerwanderungszeit \"{u}bernahmen die Goten und sp\"{a}ter dieFranken die Vorherrschaft.&
Im sp\"{a}teren Mittelalter regierten F\"{u}rsten einen Fleckenteppichvon Einzelstaaten.\\\hline
\end{tabular}
