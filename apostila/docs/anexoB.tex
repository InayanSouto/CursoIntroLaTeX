%%%%%%%%%%%%%%%%%%%%%%%%%%%%%%%%%%%%%%%%%%%%%%%%%%%%%%
%Anexo
%Este anexo foi incluindo para explicar como incluir um anexo no estilo, não existia no original desta tese.
%%%%%%%%%%%%%%%%%%%%%%%%%%%%%%%%%%%%%%%%%%%%%%%%%%%%%%%%%%%%%%%%%%%%%%%%%%%%%%%%%
\renewcommand{\thechapter}{}%
\chapter{ANEXO B - MATEMÁTICA E OUTROS SÍMBOLOS} %% Título do anexo sempre em maiúsculas. Trocar A por B no próximo anexo e etc
\label{anexoB} %% Rótulo aplicado caso queira referir-se a este tópico em qualquer lugar do texto. Trocar A por B no próximo anexo e etc
\renewcommand{\thechapter}{B}%		% trocar A por B no próximo anexo e etc

As tabelas de símbolos e entes matemáticos contidos neste anexo, foram originalmente preparadas por L. Kocbach, baseadas no documento original de David Carlisle da Universidade de Manchester e foram incorporadas a este documento para a conveniência do leitor. O documento \LaTeX{} original contendo estas tabelas, pode ser obtido em \url{http://web.ift.uib.no/Teori/KURS/WRK/TeX/symALL.html}.

% REF: http://web.ift.uib.no/Teori/KURS/WRK/TeX/symALL.html

% Math-mode symbol & verbatim
\def\W#1#2{$#1{#2}$ &\tt\string#1\string{#2\string}}
\def\X#1{$#1$ &\tt\string#1}
\def\Y#1{$\big#1$ &\tt\string#1}
\def\Z#1{\tt\string#1}

% A non-floating table environment.
%\makeatletter
%\renewenvironment{table}%
%   {\vskip\intextsep\parskip\z@
%    \vbox\bgroup\centering\def\@captype{table}}%
%   {\egroup\vskip\intextsep}
%\makeatother

% All the tables are \label'ed in case this document ever gets some
% explanatory text written, however there are no \refs as yet. To save
% LaTeX-ing the file twice we go:
%\renewcommand{\label}[1]{}

\begin{table}[H]
	\centering
	\caption{Alfabeto Grego (Maiúsculas e Minúsculas)}
	\label{tab:alfa_grego}
	%\begin{tabular}{*8l}
	\begin{tabular}{p{0.5cm} p{2.25cm} p{0.5cm} p{2.25cm} p{0.5cm} p{2.25cm} p{0.5cm} p{2.25cm}}
		\toprule
		\X\alpha        &\X\theta       &\X o           &\X\tau         \\[0.5em]
		\X\beta         &\X\vartheta    &\X\pi          &\X\upsilon     \\[0.5em]
		\X\gamma        &\X\gamma       &\X\varpi       &\X\phi         \\[0.5em]
		\X\delta        &\X\kappa       &\X\rho         &\X\varphi      \\[0.5em]
		\X\epsilon      &\X\lambda      &\X\varrho      &\X\chi         \\[0.5em]
		\X\varepsilon   &\X\mu          &\X\sigma       &\X\psi         \\[0.5em]
		\X\zeta         &\X\nu          &\X\varsigma    &\X\omega       \\[0.5em]
		\X\eta          &\X\xi                                          \\[0.5em]
		\\[0.5em]
		\X\Gamma        &\X\Lambda      &\X\Sigma       &\X\Psi         \\[0.5em]
		\X\Delta        &\X\Xi          &\X\Upsilon     &\X\Omega       \\[0.5em]
		\X\Theta        &\X\Pi          &\X\Phi \\
		\bottomrule
	\end{tabular}
\FONTE{Adaptado de \url{web.ift.uib.no/Teori/KURS/WRK/TeX/symALL.html}.}
\end{table}

\begin{table}[H]
\centering
\caption{Símbolos de Operações Binárias}
\label{tab:oper_bin}
%\begin{tabular}{*8l}
\begin{tabular}{p{0.5cm} p{2.25cm} p{0.5cm} p{2.25cm} p{0.5cm} p{3.25cm} p{0.5cm} p{2cm}}
\toprule
\X\pm           &\X\cap         &\X\diamond             &\X\oplus     \\[0.5em]
\X\mp           &\X\cup         &\X\bigtriangleup       &\X\ominus    \\[0.5em]
\X\times        &\X\uplus       &\X\bigtriangledown     &\X\otimes    \\[0.5em]
\X\div          &\X\sqcap       &\X\triangleleft        &\X\oslash    \\[0.5em]
\X\ast          &\X\sqcup       &\X\triangleright       &\X\odot      \\[0.5em]
\X\star         &\X\vee         &\X\lhd$^b$             &\X\bigcirc   \\[0.5em]
\X\circ         &\X\wedge       &\X\rhd$^b$             &\X\dagger    \\[0.5em]
\X\bullet       &\X\setminus    &\X\unlhd$^b$           &\X\ddagger   \\[0.5em]
\X\cdot         &\X\wr          &\X\unrhd$^b$           &\X\amalg     \\[0.5em]
\X+             &\X- \\
\bottomrule
\end{tabular}

%$^b$ Not predefined in a format based on {\tt basefont.tex}.
%     Use one of the style options\\
%     {\tt oldlfont}, {\tt newlfont}, {\tt amsfonts} or {\tt amssymb}.

$^b$ Símbolo não definido no {\tt basefont.tex}. Utilize uma das opções de estilos {\tt oldlfont}, {\tt newlfont}, {\tt amsfonts} ou {\tt amssymb}.

\FONTE{Adaptado de \url{web.ift.uib.no/Teori/KURS/WRK/TeX/symALL.html}.}
\end{table}

\begin{table}[H]
\centering
\caption{Símbolos Relacionais}
\label{tab:simb_rel}
%\begin{tabular}{*8l}
\begin{tabular}{p{0.5cm} p{2.25cm} p{0.5cm} p{2.25cm} p{0.5cm} p{2.25cm} p{0.5cm} p{2.25cm}}
\toprule
\X\leq          &\X\geq         &\X\equiv       &\X\models      \\[0.5em]
\X\prec         &\X\succ        &\X\sim         &\X\perp        \\[0.5em]
\X\preceq       &\X\succeq      &\X\simeq       &\X\mid         \\[0.5em]
\X\ll           &\X\gg          &\X\asymp       &\X\parallel    \\[0.5em]
\X\subset       &\X\supset      &\X\approx      &\X\bowtie      \\[0.5em]
\X\subseteq     &\X\supseteq    &\X\cong        &\X\Join$^b$    \\[0.5em]
\X\sqsubset$^b$ &\X\sqsupset$^b$&\X\neq         &\X\smile       \\[0.5em]
\X\sqsubseteq   &\X\sqsupseteq  &\X\doteq       &\X\frown       \\[0.5em]
\X\in           &\X\ni          &\X\propto      &\X=            \\[0.5em]
\X\vdash        &\X\dashv       &\X<            &\X>            \\[0.5em]
\X: \\
\bottomrule
\end{tabular}

%$^b$ Not predefined in a format based on {\tt basefont.tex}.
%     Use one of the style options\\
%     {\tt oldlfont}, {\tt newlfont}, {\tt amsfonts} or {\tt amssymb}.

$^b$ Símbolo não definido no {\tt basefont.tex}. Utilize uma das opções de estilos {\tt oldlfont}, {\tt newlfont}, {\tt amsfonts} ou {\tt amssymb}.

\FONTE{Adaptado de \url{web.ift.uib.no/Teori/KURS/WRK/TeX/symALL.html}.}
\end{table}

\begin{table}[H]
\centering
\caption{Símbolos de Pontuação Ortográfica}
\label{tab:simb_punct}
\begin{tabular}{*{5}{lp{4.5em}}}
%\begin{tabular}{p{0.5cm} p{1.25cm} p{0.5cm} p{1.25cm} p{0.5cm} p{1.25cm} p{0.5cm} p{1.25cm} p{0.5cm} p{1.25cm} p{0.5cm} p{1.25cm}}
\toprule
\X,     &\X;    &\X\colon       &\X\ldotp       &\X\cdotp \\[0.5em]
\bottomrule
\end{tabular}
\FONTE{Adaptado de \url{web.ift.uib.no/Teori/KURS/WRK/TeX/symALL.html}.}
\end{table}

\begin{table}[H]
\centering
\caption{Setas e Flechas}
\label{tab:set_flec}
%\begin{tabular}{*6l}
\begin{tabular}{p{0.5cm} p{3.75cm} p{0.5cm} p{4cm} p{0.5cm} p{3cm}}
\toprule
\X\leftarrow            &\X\longleftarrow       &\X\uparrow     \\[0.5em]
\X\Leftarrow            &\X\Longleftarrow       &\X\Uparrow     \\[0.5em]
\X\rightarrow           &\X\longrightarrow      &\X\downarrow   \\[0.5em]
\X\Rightarrow           &\X\Longrightarrow      &\X\Downarrow   \\[0.5em]
\X\leftrightarrow       &\X\longleftrightarrow  &\X\updownarrow \\[0.5em]
\X\Leftrightarrow       &\X\Longleftrightarrow  &\X\Updownarrow \\[0.5em]
\X\mapsto               &\X\longmapsto          &\X\nearrow     \\[0.5em]
\X\hookleftarrow        &\X\hookrightarrow      &\X\searrow     \\[0.5em]
\X\leftharpoonup        &\X\rightharpoonup      &\X\swarrow     \\[0.5em]
\X\leftharpoondown      &\X\rightharpoondown    &\X\nwarrow     \\[0.5em]
\X\rightleftharpoons    &\X\leadsto$^b$ \\
\bottomrule
\end{tabular}

%$^b$ Not predefined in a format based on {\tt basefont.tex}.
%     Use one of the style options\\
%     {\tt oldlfont}, {\tt newlfont}, {\tt amsfonts} or {\tt amssymb}.

$^b$ Símbolo não definido no {\tt basefont.tex}. Utilize uma das opções de estilos {\tt oldlfont}, {\tt newlfont}, {\tt amsfonts} ou {\tt amssymb}.

\FONTE{Adaptado de \url{web.ift.uib.no/Teori/KURS/WRK/TeX/symALL.html}.}
\end{table}

\begin{table}[H]
\centering
\caption{Outros Símbolos}
\label{tab:outros_simbs}
%\begin{tabular}{*8l}
\begin{tabular}{p{0.5cm} p{2.25cm} p{0.5cm} p{2.25cm} p{0.5cm} p{2.25cm} p{0.5cm} p{2.25cm}}
\toprule
\X\ldots        &\X\cdots       &\X\vdots       &\X\ddots       \\[0.5em]
\X\aleph        &\X\prime       &\X\forall      &\X\infty       \\[0.5em]
\X\hbar         &\X\emptyset    &\X\exists      &\X\Box$^b$     \\[0.5em]
\X\imath        &\X\nabla       &\X\neg         &\X\Diamond$^b$ \\[0.5em]
\X\jmath        &\X\surd        &\X\flat        &\X\triangle    \\[0.5em]
\X\ell          &\X\top         &\X\natural     &\X\clubsuit    \\[0.5em]
\X\wp           &\X\bot         &\X\sharp       &\X\diamondsuit \\[0.5em]
\X\Re           &\X\|           &\X\backslash   &\X\heartsuit   \\[0.5em]
\X\Im           &\X\angle       &\X\partial     &\X\spadesuit   \\[0.5em]
\X\mho$^b$      &\X.            &\X| \\
\bottomrule
\end{tabular}

%$^b$ Not predefined in a format based on {\tt basefont.tex}.
%     Use one of the style options\\
%     {\tt oldlfont}, {\tt newlfont}, {\tt amsfonts} or {\tt amssymb}.

$^b$ Símbolo não definido no {\tt basefont.tex}. Utilize uma das opções de estilos {\tt oldlfont}, {\tt newlfont}, {\tt amsfonts} ou {\tt amssymb}.

\FONTE{Adaptado de \url{web.ift.uib.no/Teori/KURS/WRK/TeX/symALL.html}.}
\end{table}

\begin{table}[H]
\centering
\caption{Símbolos na Escala das Variáveis}
\label{tab:simb_esc_var}
%\begin{tabular}{*6l}
\begin{tabular}{p{0.5cm} p{3.75cm} p{0.5cm} p{4cm} p{0.5cm} p{3cm}}
\toprule
\X\sum          &\X\bigcap      &\X\bigodot     \\[0.5em]
\X\prod         &\X\bigcup      &\X\bigotimes   \\[0.5em]
\X\coprod       &\X\bigsqcup    &\X\bigoplus    \\[0.5em]
\X\int          &\X\bigvee      &\X\biguplus    \\[0.5em]
\X\oint         &\X\bigwedge \\
\bottomrule
\end{tabular}
\FONTE{Adaptado de \url{web.ift.uib.no/Teori/KURS/WRK/TeX/symALL.html}.}
\end{table}

\begin{table}[H]
\centering
\caption{Símbolos Logarítmicos e Trigonométricos}
\label{tab:sim_log_trig}
%\begin{tabular}{*8l}
\begin{tabular}{p{1.6cm} p{1.5cm} p{1.5cm} p{1.5cm} p{1.5cm} p{1.5cm} p{1.5cm} p{1.5cm}}
\toprule
\Z\arccos &\Z\cos  &\Z\csc &\Z\exp & \Z\ker    &\Z\limsup &\Z\min &\Z\sinh \\[0.5em]
\Z\arcsin &\Z\cosh &\Z\deg &\Z\gcd & \Z\lg     &\Z\ln     &\Z\Pr  &\Z\sup  \\[0.5em]
\Z\arctan &\Z\cot  &\Z\det &\Z\hom & \Z\lim    &\Z\log    &\Z\sec &\Z\tan  \\[0.5em]
\Z\arg    &\Z\coth &\Z\dim &\Z\inf & \Z\liminf &\Z\max    &\Z\sin &\Z\tanh \\
\bottomrule
\end{tabular}
\FONTE{Adaptado de \url{web.ift.uib.no/Teori/KURS/WRK/TeX/symALL.html}.}
\end{table}

\begin{table}[H]
\centering
\caption{Delimitadores}
\label{tab:delimitadores}
%\begin{tabular}{*8l}
\begin{tabular}{p{0.5cm} p{2cm} p{0.5cm} p{2cm} p{0.5cm} p{2.5cm} p{0.5cm} p{2.25cm}}
\toprule
\X(             &\X)            &\X\uparrow     &\X\Uparrow     \\[0.5em]
\X[             &\X]            &\X\downarrow   &\X\Downarrow   \\[0.5em]
\X\{            &\X\}           &\X\updownarrow &\X\Updownarrow \\[0.5em]
\X\lfloor       &\X\rfloor      &\X\lceil       &\X\rceil       \\[0.5em]
\X\langle       &\X\rangle      &\X/            &\X\backslash   \\[0.5em]
\X|             &\X\| \\
\bottomrule
\end{tabular}
\FONTE{Adaptado de \url{web.ift.uib.no/Teori/KURS/WRK/TeX/symALL.html}.}
\end{table}

\begin{table}[H]
\centering
\caption{Delimitadores Grandes}
\label{tab:ldels}
%\begin{tabular}{*8l}
\begin{tabular}{p{0.5cm} p{2.25cm} p{0.5cm} p{2.25cm} p{0.5cm} p{2.25cm} p{0.5cm} p{2.25cm}}
\toprule
\Y\rmoustache&  \Y\lmoustache&  \Y\rgroup&      \Y\lgroup\\[5pt]
\Y\arrowvert&   \Y\Arrowvert&   \Y\bracevert \\
\bottomrule
\end{tabular}
\FONTE{Adaptado de \url{web.ift.uib.no/Teori/KURS/WRK/TeX/symALL.html}.}
\end{table}

\begin{table}[H]
\centering
\caption{Acentos Matemáticos}
\label{tab:acentos}
\begin{tabular}{*{10}l}
\toprule
\W\hat{a}     &\W\acute{a}  &\W\bar{a}    &\W\dot{a}    &\W\breve{a}\\[0.5em]
\W\check{a}   &\W\grave{a}  &\W\vec{a}    &\W\ddot{a}   &\W\tilde{a}\\
\bottomrule
\end{tabular}
\FONTE{Adaptado de \url{web.ift.uib.no/Teori/KURS/WRK/TeX/symALL.html}.}
\end{table}

\begin{table}[H]
\centering
\caption{Algumas Outras Construções}
\label{tab:outos_simbs}
%\begin{tabular}{*4l}
\begin{tabular}{p{1cm} p{5cm} p{1cm} p{5cm}}
\toprule
\W\widetilde{abc}       &\W\widehat{abc}                          \\[0.5em]
\W\overleftarrow{abc}   &\W\overrightarrow{abc}                   \\[0.5em]
\W\overline{abc}        &\W\underline{abc}                        \\[0.5em]
\W\overbrace{abc}       &\W\underbrace{abc}                       \\[5pt]
\W\sqrt{abc}            &$\sqrt[n]{abc}$&\verb|\sqrt[n]{abc}|     \\[0.5em]
$f'$&\verb|f'|          &$\frac{abc}{xyz}$&\verb|\frac{abc}{xyz}| \\
\bottomrule
\end{tabular}
\FONTE{Adaptado de \url{web.ift.uib.no/Teori/KURS/WRK/TeX/symALL.html}.}
\end{table}
