% Adaptado de http://www.texample.net/tikz/examples/tex-workflow/
% carlos.bastarz@inpe.br (28/09/2019)

\documentclass{article}

\usepackage[utf8]{inputenc}
\usepackage{tikz}

\usetikzlibrary{arrows,shapes}

\usepackage{lscape}

\begin{document}

\begin{landscape}

	\thispagestyle{empty}
	
	\tikzstyle{format} = [draw, thin, fill=blue!20]
	\tikzstyle{medium} = [ellipse, draw, thin, fill=green!20, minimum height=2.5em]
	
	\begin{figure}
		\resizebox{\columnwidth}{!}{%
			\begin{tikzpicture}[node distance=3cm, auto,>=latex', thick]
			    \path[use as bounding box] (-1,0) rectangle (10,-2);
			    \path[->] node[format] (tex) {arq. {\tt .tex}};
			    \path[->] node[format, right of=tex] (dvi) {arq. {\tt .dvi}}
			                  (tex) edge node {\LaTeX} (dvi);
			    \path[->] node[format, right of=dvi] (ps) {arq. {\tt .ps}}
			                  node[medium, below of=dvi] (screen) {visualização}
			                  (dvi) edge node {dvips} (ps)
			                        edge node[swap] {xdvi} (screen);
			    \path[->] node[format, right of=ps] (pdf) {arq. {\tt .pdf}}
			                  node[medium, below of=ps] (print) {impressão}
			                  (ps) edge node {ps2pdf} (pdf)
			                       edge node[swap] {gs} (screen)
			                       edge (print);
			    \path[->] (pdf) edge (screen)
			                        edge (print);
			    \path[->, draw] (tex) -- +(0,1) -| node[near start] {pdf\LaTeX} (pdf);
			\end{tikzpicture}
		}
	\end{figure}

\end{landscape}

\end{document}
