%%%%%%%%%%%%%%%%%%%%%%%%%%%%%%%%%%%%%%%%%%%%%%%%%%%%%%%%%%%%%%%%%%%%%%%%%%%%%%%%
% ABSTRACT


\begin{abstract}

%% neste arquivo abstract.tex
%% o texto do resumo e as palavras-chave têm que ser em Inglês para os documentos escritos em Português
%% o texto do resumo e as palavras-chave têm que ser em Português para os documentos escritos em Inglês
%% os nomes dos comandos \begin{abstract}, \end{abstract}, \keywords e \palavrachave não devem ser alterados

\selectlanguage{english}	%% para os documentos escritos em Português
%\selectlanguage{portuguese}	%% para os documentos escritos em Inglês

\hypertarget{estilo:abstract}{} %% uso para este Guia

%In this work the possible chaotic nature of the atmospheric turbulence is analysed. The analyses carried out here, based in data of high resolution temperature, obtained from the WETAMC campaign of the LBA project, suggest the existence of a low-dimension chaotic behavior in the atmospheric boundary layer. The corresponding chaotic attractor possess a correlation dimension of $D_{2}=3.50\pm0.05$. The presence of chaotic dynamics in the analysed data is confirmed with the estimate of a small Lyapunov exponent but positive, with value $\lambda_{1}=0.050\pm0.002$. However, this low-dimension chaotic dynamics is associated with the presence of the coherent structures in the atmospheric boundary layer and not to the atmospheric turbulence. This affirmation is evidenced by the process of filtering for wavelets used in the studied experimental data, that allow to separate the contribution of the coherent structures of the turbulent background signal. 

%Esta apostila apresenta a linguagem de marcação \LaTeX{} para a confecção de textos científicos, tendo como foco o estilo de publicações do INPE. São apresentados os aspectos históricos de formulação da linguagem e as motivações para o seu uso dentro do ambiente acadêmico. O objetivo principal do uso da linguagem é permitir que o usuário concentre-se na escrita do texto, no desenvolvimento das suas ideias sem ter que se preocupar com a determinação e o posicionamento dos diversos elementos estruturais de um documento.  Não se trata, porém, de um curso de escrita científica, mas sim de um manual objetivo para a aplicação da linguagem de marcação \LaTeX{}. Ao final, o usuário estará apto a aplicar a linguagem no desenvolvimento dos seus trabalhos.

This material presents the \LaTeX{} markup languagem as a type writing tools focused at the INPE's style sheet. It is presented the historical aspects of the language formulation as well as the motivations for its use in the academia. The iams for the languagem usage is to allow the user to keep focused in the writing and the ideas development without the need to focus at textual elements. It is not a scientific writing course though, but a objective manual for the language application. At the end, it is expected that the user is able to apply the languagem markup system at the type writing of their own documents.

\keywords{%
	\palavrachave{\LaTeX{}}%
	\palavrachave{Scientific Writing}%
	\palavrachave{Markup Language}%
}

\selectlanguage{portuguese}	%% para os documentos escritos em Português
%\selectlanguage{english}	%% para os documentos escritos em Inglês

\end{abstract}