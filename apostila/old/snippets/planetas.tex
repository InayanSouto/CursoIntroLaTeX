%%%%%%%%%%%%%%%%%%%%%%%%%%%%%%%%%%%%%%%%%%%%%%%%%%%%%%%%%%%%%%%
%
% Welcome to Overleaf --- just edit your LaTeX on the left,
% and we'll compile it for you on the right. If you open the
% 'Share' menu, you can invite other users to edit at the same
% time. See www.overleaf.com/learn for more info. Enjoy!
%
%%%%%%%%%%%%%%%%%%%%%%%%%%%%%%%%%%%%%%%%%%%%%%%%%%%%%%%%%%%%%%%
% Planets
% Author: Gerard Fleuter
%\documentclass[tikz,border=10pt]{standalone}
%%%<
%\usepackage{verbatim}
%%%>
\begin{comment}
:Title: Planets
:Tags: Styles;Physics;Astronomy
:Author: Gerard Fleuter
:Slug: planets

Planets in our solar system.
\end{comment}
%\begin{document}
% definition de partial ellipse
\tikzset{partial ellipse/.style args =
  {#1:#2:#3}{insert path={+ (#1:#3) arc (#1:#2:#3)}}}
\begin{tikzpicture}[>=latex]
  %  ellipses
  \draw [fill=white!90!red]    (3,-1.8) ellipse    (4cm and 1 cm);
  \draw [fill=yellow!90!green] (3,-1.8) ellipse (3cm and 0.75 cm);
  \draw [fill=white!90!green]  (3,-1.8) ellipse  (2cm and 0.5 cm);

  % -- Soleil
  \shade [ball color=gray!10!yellow] (3,-1.8) circle (1);
  \node (soleil) at (3,-1.8) {\bf Sol};
  % partial ellipse pour tracé devant le Soleil
  \draw (3,-1.8) [partial ellipse=220:320:2cm and 0.5cm]
        (3,-1.8) [partial ellipse=220:320:3cm and 0.75cm];

  % Venus
  \shade [ball color=gray!10!orange] (1.6,-1.8) circle (.2);
  \node (venus) at (1.5,-1.45) {Vênus}; 

  % ombre de Venus
  \draw[color=white!70!black,fill=white!70!black]
    (1.6,-2.3) ellipse (2mm and 0.5mm);

  % Mercure
  \shade [ball color=gray!10!orange] (5,-1.225) circle (.25);
  \node (mercure) at (5,-0.8) {Mercúrio}; 

  % Earth
  \shade [ball color=white!50!blue] (5.75,-2.5) circle (.33);
  \node (terre) at (6.6,-2.6) {\bf Terra};

  % Lune
  \shade [ball color=yellow] (5.25,-2.8) circle (.1);
  \node (lune) at (5.25,-3) {Lua};
     
  % Mars
  \draw (3,-1.8) [partial ellipse=45:120:9cm and 2.5cm];
  \shade [ball color=black!50!red] (5,0.66) circle (.15);
  \node (mars) at (5,1) {\bf Marte};   
  % trajet
  \draw [line width=2pt,blue,->,>=latex] (terre) to[out=0,in=0] (mars);   
\end{tikzpicture}
%\end{document}