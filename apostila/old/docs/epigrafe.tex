%%%%%%%%%%%%%%%%%%%%%%%%%%%%%%%%%%%%%%%%%%%%%%%%%%%%%%%%%%%%%%%%%%%%%%%%%%%%%%%%
% Epígrafe %% opcional

\begin{epigrafe} %% insira sua epígrafe abaixo; estilo livre

\hypertarget{estilo:epigrafe}{} %% uso para este Guia
 
\textit{\large``The language in which we express our ideas has a strong influence on our thought processes.''}

\vspace{1cm}

\hspace{4cm} \emph{\textsc{Donald Ervin Knuth}}\\\hspace{4cm} em \textsl{``Literate Programming''}, 1992

%For example, if \lstinline|\thinmskip = 3mu|, this makes \lstinline|\thickmskip = 6mu|. But if you also want to use \lstinline|\skip12| for horizontal glue, whether in math mode or not, the amount of skipping will be in points (e.g., 6pt). The rule is that glue in math mode varies with the size only when it is an \lstinline|\mskip|; when moving between an mskip and ordinary skip, the conversion factor \lstinline|1mu=1pt| is always used. The meaning of `\lstinline|\mskip\skip12|' and `\lstinline|\baselineskip=\the\thickmskip\lstinline|' should be clear.
%
%\vspace{1cm}
%
%\hspace{4cm} \emph{\textsc{Donald Ervin Knuth}}\\\hspace{4cm} em \textsl{``\TeX 82 -- Comparison with \TeX80''}

\end{epigrafe}