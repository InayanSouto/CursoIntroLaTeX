\documentclass[17pt]{extarticle}
\usepackage[utf8]{inputenc}
\usepackage{lipsum}
\usepackage{extsizes}

\title{Título}
\author{Nome}
\date{\today}

\begin{document}

\maketitle

A Equação \ref{eq:euler} é denominada ``Equação de Euler'' e nela estão relacionados os números irracionais mais conhecidos: $e$ e $\pi$, além do número imaginário $i$:

\begin{equation}
  \label{eq:euler}
  e^{i\pi} + 1 = 0
\end{equation}

A série de MacLaurin para $e^{x}$ é dada por:
\begin{equation}
  \label{eq:maclaurin}
  e^{x} = \sum_{k=0}^{\infty} \frac{x^{k}}{k!}
\end{equation}

\newpage

Na página \pageref{eq:euler} foi apresentada uma equação que conjuga os números irreacionais $e$ e $\pi$ e o número imaginário $i$.

A série de MacLaurin para $e^{x}$ está expressa na página \pageref{eq:maclaurin}.

\end{document}
