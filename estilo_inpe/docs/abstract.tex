%%%%%%%%%%%%%%%%%%%%%%%%%%%%%%%%%%%%%%%%%%%%%%%%%%%%%%%%%%%%%%%%%%%%%%%%%%%%%%%%
% ABSTRACT


\begin{abstract}

%% neste arquivo abstract.tex
%% o texto do resumo e as palavras-chave têm que ser em Inglês para os documentos escritos em Português
%% o texto do resumo e as palavras-chave têm que ser em Português para os documentos escritos em Inglês
%% os nomes dos comandos \begin{abstract}, \end{abstract}, \keywords e \palavrachave não devem ser alterados

\selectlanguage{english}	%% para os documentos escritos em Português
%\selectlanguage{portuguese}	%% para os documentos escritos em Inglês

\hypertarget{estilo:abstract}{} %% uso para este Guia

In this work the possible chaotic nature of the atmospheric turbulence is analysed. The analyses carried out here, based in data of high resolution temperature, obtained from the WETAMC campaign of the LBA project, suggest the existence of a low-dimension chaotic behavior in the atmospheric boundary layer. The corresponding chaotic attractor possess a correlation dimension of $D_{2}=3.50\pm0.05$. The presence of chaotic dynamics in the analysed data is confirmed with the estimate of a small Lyapunov exponent but positive, with value $\lambda_{1}=0.050\pm0.002$. However, this low-dimension chaotic dynamics is associated with the presence of the coherent structures in the atmospheric boundary layer and not to the atmospheric turbulence. This affirmation is evidenced by the process of filtering for wavelets used in the studied experimental data, that allow to separate the contribution of the coherent structures of the turbulent background signal. 

\keywords{%
	\palavrachave{Atmospheric turbulence}%
	\palavrachave{WETAMC campaign}%
	\palavrachave{LBA project}%
	\palavrachave{Chaotic behavior}%
	\palavrachave{Chaotic attractor}%
}

\selectlanguage{portuguese}	%% para os documentos escritos em Português
%\selectlanguage{english}	%% para os documentos escritos em Inglês

\end{abstract}
